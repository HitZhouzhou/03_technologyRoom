# Requests库入门
## Requests库的7个方法:
- request(): 构造一个请求,支撑以下各方法的基础方法
- get(): 获取HTML网页的主要方法,对应于HTTP的GET
- head(): 获取HTML网页头文件的方法,对应于HTTP的HEAD
- post(): 向HTML网页提交POST请求的方法,对应于HTTP的POST
- put(): 向HTML网页提交PUT请求的方法,对应于HTTP的PUT
- patch(): 向HTML网页提交局部修改请求,对应于HTTP的PATCH
- delete(): 向HTML页面提交删除请求,对应于HTTP的DELETE

## HTTP 协议
HTTP , Hypertext Transfer Protocol,超文本传输协议.
HTTP协议是一个基于"请求与响应"模式的,无状态的应用层协议

URL格式: http://host[:port][path]
host: 合法的Internet主机域名或IP地址
port: 端口号,缺省端口为80
path: 请求资源的路径

例子:http://www.bit.edu.cn/  http://www.bit.edu.cn/xxx/xxx.html
URL 是通过HTTP协议存取资源的Internet路径,一个URL对应一个数据资源

### HTTP请求方法
- GET: 请求获取URL位置的资源
- HEAD: 请求获取URL位置资源的响应消息报头,即获得该资源的头部信息
- POST: 请求向URL位置的资源后附加新的数据
- PUT: 请求向URL位置存储一个资源,覆盖原URL位置的资源
- PATCH: 请求局部更新URL位置的资源,即改变该处资源的部分内容
- DELETE: 请求删除URL位置存储的资源

### HTTP协议对资源的操作
- GET: 获取资源
- HEAD: 获取资源的首部
- POST: 传输资源
- PUT: 更新资源
- DELETE: 删除资源
- TRACE: 追踪资源
(在这里补充一个图片)

### 理解PATCH和PUT的区别
假设URL位置有一组数据UserInfo,包括UserID,UserName等20个字段,我通过POST方法提交一个更新用户信息的请求,其中仅包含UserName和UserAge字段,其他字段不变.此时,若使用PUT方法,会对URL位置的UserInfo数据全部更新,没有提交的字段会被删除.若使用PATCH方法,只会更新UserName和UserAge字段,其他字段不变

PATCH 的好处:节省网络带宽

